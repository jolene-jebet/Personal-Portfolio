\documentclass[11pt,a4paper,sans,swahili]{moderncv}        
\moderncvstyle{casual}                             
\moderncvcolor{blue}                               
\usepackage[utf8]{inputenc}                       
\usepackage[scale=0.75,a4paper]{geometry}
\usepackage{babel}
\usepackage{comment}

%----------------------------------------------------------------------------------
%            personal data
%----------------------------------------------------------------------------------
\firstname{Gathu }
\familyname{Macharia}
\title{Mtaalamu wa Takwimu}                              
\address{Barabara ya Nyeri-Mweiga}{S.L.P 657-10100 Nyeri}{Kenya}         
\mobile{+254702 499 673}                          
\email{sgathuh@gmail.com}                               
\homepage{In Progress}                         
\quote{Nadharia ina mbadala mbili tu: kuwa sahihi au kuwa si sahihi. Mfano una uwezekano wa tatu: inaweza kuwa sahihi, lakini isiyohusiana\\
	\textit{\textbf{Jadish Mehra}}}                                
\social[linkedin]{gathu-simon-82a275199/}
\social[github]{Gathuh}

\begin{document}
	
	\makecvtitle
	
	\section{Elimu}
	
	\cventry{2021--2024}{Shahada ya Sayansi katika Takwimu na Sayansi ya Bima}{Chuo Kikuu cha Teknolojia cha Dedan Kimathi}{Nyeri}{}{Takwimu na Uundaji wa Takwimu kwa kutumia R na Python}  
	\cventry{2024}{Shule ya Majira ya joto ya Sayansi ya Takwimu na Warsha}{Data Science Africa}{Nyeri}{}{Sayansi ya Takwimu kwa ajili ya manufaa ya jamii, Maono ya Kompyuta, Ukusanyaji wa Data, Uchimbaji wa vipengele na Uundaji wa Takwimu}
	\cventry{2021--2021}{Shahada ya Mitandao ya Cisco}{Chuo Kikuu cha Kilimo na Teknolojia cha Jomo Kenyatta}{Kiambu}{}{Mitandao ya Msingi na Taswira ya Mtandao}
	
	\cventry{2016--2019}{Cheti cha Elimu ya Sekondari Kenya}{Utumishi Academy}{Gilgil}{}{Cheti cha Sekondari na Msingi wa Kompyuta}
	
	\cventry{2008--2015}{Cheti cha Elimu ya Msingi Kenya}{Shule ya Msingi ya Complex}{Mumias}{}{Cheti cha Msingi}
	
	\section{Tasnifu ya Shahada}
	\cvitem{Kichwa}{Uundaji wa Muda wa Kupona kwa Bei za Hisa kwa kutumia Mfano wa AFT}
	\cvitem{Wanaosimamia}{Dr. Mundia}
	\cvitem{Maelezo}{Kutumia mifano ya kudumu ya kitakwimu kuunda muda wa kupona kwa hisa za kampuni za Kenya. Hatukuweza kutumia viwango vya hatari visivyolingana hivyo tukatumia mfano wa hatari zisizolingana ambao ni AFT}
	
	\section{Uzoefu}
	\subsection{Kazi za Likizo}
	\cventry{2023--2024}{Usimamizi wa Hatari na Ukaguzi wa Ndani}{Mfuko wa Hifadhi ya Jamii}{Nairobi}{}{Kutumia sheria ya Benford kugundua udanganyifu. Kutumia programu za IDEA, R, na Excel kuchambua data za ukaguzi. Pia, kutengeneza faili za kufuata sheria na kuandika ripoti za shamba husika.}
	
	\cventry{2022--Sasa}{Kazi za Kujitegemea}{Freelancer}{Nyeri}{}{Kutumia R, SPSS na Python kufanya majaribio ya mawazo, mifano ya jumla, mifano ya Bayesian, Mifano ya Kuongeza Gradients, Mitandao ya Neva na mingine.}
	
	\subsection{Mbalimbali}
	\cventry{2019--2021}{Duka la Kuchapisha}{Fastnet Cyber}{Nanyuki}{}{Kujifunza jinsi ya kushughulikia nyaraka, Kuchapisha, Kifungashio.}
	
	\section{Lugha}
	\cvitemwithcomment{Kiswahili}{Mtaalamu}{Nafurahia} 
	\cvitemwithcomment{Kiingereza}{Mtaalamu}{Kusoma}
	\cvitemwithcomment{Gikuyu}{Mtaalamu}{Kuzungumza}
	
	\section{Ujuzi}
	\cvdoubleitem{R}{Utaalamu katika Uprogramu wa Takwimu}{Mtaalamu}{Utaswira wa Data}
	\cvdoubleitem{Python}{Utaalamu katika Ujifunzaji wa Mashine}{Nzuri}{Ukusanyaji wa Tovuti}
	\cvdoubleitem{SPSS}{Kuthibitisha Takwimu}{Kategoria 6}{Maoni}
	
	\section{Shughuli za ziada}
	\cvitem{Kuangalia Hati za Video}{Maelezo}
	\cvitem{Matembezi ya Asili}{Maelezo}
	\cvitem{Hobby 3}{Maelezo}
	
	\section{Miradi}
	\cvlistitem{Kutabiri bei za hisa za kampuni zilizoorodheshwa NSE kwa kutumia GARCH Extensions.}
	\cvlistitem{Kujenga mfano wa utabiri kwa kutumia regression ya logisitic kutathmini uwezekano wa mteja kuacha.}
	
	\section{Shughuli za Ziada}
	\cvlistitem{Mwanachama wa Data Science Africa}
	\cvlistitem{Kushiriki katika Mashindano ya Kitaifa ya 2021 katika UON na kupata ushindi.}
	\cvlistitem{Klabu ya Sayansi ya Bima}
	\cvlistitem{Jumuiya ya Sayansi ya Takwimu DSAIC (Mwalimu)}


	\section{Marejeleo}
	Upon Request
		\begin{comment}
	\begin{center}
		\begin{tabular}{p{6cm}p{8cm}}
			\textbf{Jina:} & Dr. Mundia S.M, Mhadhiri\\
			\textbf{Nafasi:} & Mratibu wa Miradi\\
			\textbf{Kampuni:} & Chuo Kikuu cha Teknolojia cha Dedan Kimathi \\
			\textbf{Mawasiliano:} & +254 721 302 869\\
			\textbf{Barua Pepe:} & simon.maina@dkut.ac.ke \\
			\\
			\textbf{Jina:} & Dr. Omari Cyprian \\
			\textbf{Nafasi:} & Mwenyekiti \\
			\textbf{Kampuni:} & Shahada ya Sayansi ya Takwimu na Sayansi ya Bima, Chuo Kikuu cha Teknolojia cha Dedan Kimathi \\
			\textbf{Mawasiliano:} & +254 722 616 725 \\
			\textbf{Barua Pepe:} & cyomari@dkut.ac.ke \\
			\\
			\textbf{Jina:} & Julius Miheso \\
			\textbf{Nafasi:} & Mkaguzi wa Ndani \\
			\textbf{Kampuni:} & NSSF \\
			\textbf{Mawasiliano:} & +254 720 652 682 \\
			\textbf{Barua Pepe:} & atsiaya80@gmail.com \\
			\\
			\textbf{Jina:} & Gideon Ngome \\
			\textbf{Nafasi:} & Meneja wa Hatari \\
			\textbf{Kampuni:} & NSSF \\
			\textbf{Mawasiliano:} & +254 746 595 822 \\
			\textbf{Barua Pepe:} & ngomegideon@yahoo.com \\
		\end{tabular}
	\end{center}
	\end{comment}
\end{document}
